%%%%%%%%%%%%%%%%%%%%%%%%%%%%%%%%%%%%%%%%%%%%%%%%%%%%%%%%
%                IAML 2021 Assignment 1                %
%                                                      %
%                                                      %
% Authors: Oisin Mac Aodha and Octave Mariotti         %
% Using template from: Michael P. J. Camilleri and     %
% Traiko Dinev.                                        %
%                                                      %
% Based on the Cleese Assignment Template for Students %
% from http://www.LaTeXTemplates.com.                  %
%                                                      %
% Original Author: Vel (vel@LaTeXTemplates.com)        %
%                                                      %
% License:                                             %
% CC BY-NC-SA 3.0                                      %
% (http://creativecommons.org/licenses/by-nc-sa/3.0/)  %
%                                                      %
%%%%%%%%%%%%%%%%%%%%%%%%%%%%%%%%%%%%%%%%%%%%%%%%%%%%%%%%

%--------------------------------------------------------
%   IMPORTANT: Do not touch anything in this part
\documentclass[12pt]{article}
\input{style.tex}



% Options for Formatting Output

\global\setbool{clearon}{true} %
\global\setbool{authoron}{true} %
\ifbool{authoron}{\rhead{\small{\assignmentAuthorName}}\cfoot{\small{\assignmentAuthorName}}}{\rhead{}}

\newcommand{\assignmentQuestionName}{Question}
\newcommand{\assignmentTitle}{Assignment\ \#1}

\newcommand{\assignmentClass}{IAML -- INFR10069 (LEVEL 10)}

\newcommand{\assignmentWarning}{NO LATE SUBMISSIONS} % 
\newcommand{\assignmentDueDate}{Monday,\ October\ 18,\ 2021 @ 16:00}
%--------------------------------------------------------


%%%%%%%%%%%%%%%%%%%%%%%%%%%%%%%%%%%%%%%%%%%%%%%%%%%%%%%
%
% NOTE: YOU NEED TO ENTER YOUR STUDENT ID BELOW.
%
%%%%%%%%%%%%%%%%%%%%%%%%%%%%%%%%%%%%%%%%%%%%%%%%%%%%%%%%
%--------------------------------------------------------
% IMPORTANT: Specify your Student ID below. You will need to uncomment the line, else compilation will fail. Make sure to specify your student ID correctly, otherwise we may not be able to identify your work and you will be marked as missing.
\newcommand{\assignmentAuthorName}{s1904845}
%--------------------------------------------------------

\begin{document}

%%%%%%%%%%%%%%%%%%%%%%%%%%%%%%%%%%%%%%%%%%%%%%%%%%%%%%%%%%%%%%%%%%%%%%%%%%%%%%
%============================================================================%
%%%%%%%%%%%%%%%%%%%%%%%%%%%%%%%%%%%%%%%%%%%%%%%%%%%%%%%%%%%%%%%%%%%%%%%%%%%%%%
\clearpage

\begin{question}{Linear Regression}

\questiontext{We will fit linear regression models to the data in file \texttt{regression\_part1.csv}.}



%
%
\begin{subquestion}{Describe the main properties of the data, focusing on the size, data ranges, and data types.   
}


\begin{answerbox}{10em}
We have two attributes in the dataset. For both attributes, we have 50 observations and both data type are float64.  The first attribute is revision time, which ranges from 2.723 to 48.011, with mean 22.220020. The second attribute is exam score, which ranges from 14.731 to 94.945, with mean 49.919860.
% \begin{center}
%     \includegraphics[width=0.4\textwidth]{20211018145355.png}
% \end{center}
\end{answerbox}



\end{subquestion}




%
%
\begin{subquestion}{Fit a linear model to the data so that we can predict \texttt{exam\_score} from \texttt{revision\_time}. 
Report the estimated model parameters $\mathbf{w}$. 
Describe what the parameters represent for this 1D data. 
For this part, you should use the sklearn implementation of \href{https://scikit-learn.org/0.19/modules/generated/sklearn.linear_model.LinearRegression.html}{Linear Regression}.\\
\hint{By default in sklearn \texttt{fit\_intercept = True}. Instead, set \texttt{fit\_intercept = False} and pre-pend $1$ to each value of $x_i$ yourself to create $\boldsymbol{\phi}(x_i) = [1, x_i]$. 
}
}


\begin{answerbox}{10em}
The estimated model parameters is $w = [ 17.89768026,  1.44114091]$. This result shows the student with 0 revision time will score 17.89768026 in the exam. And as the revision time increases by 1, the exam score will increase by 1.44114091.
\end{answerbox}



\end{subquestion}



%
%
\begin{subquestion}{Display the fitted linear model and the input data on the same plot.
}


\begin{answerbox}{35em}
\begin{center}
    \includegraphics[width=1\textwidth]{fitted_linear_model.jpg}
\end{center}
\end{answerbox}



\end{subquestion}



%
%
\begin{subquestion}{Instead of using sklearn, implement the closed-form solution for fitting a linear regression model yourself using numpy array operations.  
Report your code in the answer box.
It should only take a few lines (i.e. <5).\\ 
\hint{Only report the relevant lines for estimating $\mathbf{w}$ e.g. we do not need to see the data loading code. You can write the code in the answer box directly or paste in an image of it. }
}


\begin{answerbox}{20em}
\begin{verbatim}
pseudo_inverse = np.linalg.inv(np.dot(x_train.T,x_train))
pseudo_inverse = np.dot(pseudo_inverse, x_train.T)
phi_w = np.dot(pseudo_inverse, y_train)
phi_w
\end{verbatim}
\end{answerbox}



\end{subquestion}



%
%
\begin{subquestion}{Mean Squared Error (MSE) is a common metric used for evaluating the performance of regression models. 
Write out the expression for MSE and list one of its limitations. \\
\hint{For notation, you can use $y$ for the ground truth quantity and $\hat{y}$ (\texttt{\$\textbackslash{}hat\{y\}\$} in latex) in place of the model prediction.}
}


\begin{answerbox}{10em}
$MSE = \sqrt{\frac{1}{n} \sum_{i=1}^{n}(\hat{y_i}-y_i)^2  }$
\end{answerbox}



\end{subquestion}


 
%
%
\begin{subquestion}{Our next step will be to evaluate the performance of the fitted models using Mean Squared Error (MSE). 
Report the MSE of the data in \texttt{regression\_part1.csv} for your prediction of \texttt{exam\_score}.
You should report the MSE for the linear model fitted using sklearn and the model resulting from your closed-form solution. 
Comment on any differences in their performance. 
}


\begin{answerbox}{10em}
Closed-form solution has $MSE=5.5664596122258265$. Sklearn solution has $MSE=5.566459612225826$. There is only a very small difference between both solutions. Sklearn solution is slightly better compared to closed form solution. This might cause by using precise data type in my closed form solution.
\end{answerbox}



\end{subquestion}






 
\end{question}






\end{document}
